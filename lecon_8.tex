\subsection{Question 1}

\subsubsection{Point 1}

Données :

\begin{equation}
	\begin{aligned}
		\int_a^b f(x) &\approx \sum_{i=1}^n f(x^{(i)}) \cdot h \cdot \int_1^n\frac{\prod_{j \neq i}(t-j)}{\prod_{j \neq i} (i - j)} dt\\
		n &= 3\\
		h &= \frac{b-a}{n-1} = \frac{b-a}{2}
	\end{aligned}
\end{equation}

Résolution :

\begin{equation}
	\begin{aligned}
		\int_a^b f(x) &\approx \sum_{i=1}^3 f(x^{(i)}) \cdot \frac{b-a}{2} \cdot \int_1^3\frac{\prod_{j \neq i}(t-j)}{\prod_{j \neq i} (i - j)} dt\\
		&= f(x^{(1)}) \cdot \frac{b-a}{2} \cdot \omega_1 + f(x^{(2)}) \cdot \frac{b-a}{2} \cdot \omega_2 + f(x^{(3)}) \cdot \frac{b-a}{2} \cdot \omega_3
	\end{aligned}
\end{equation}

On peut calculer les $\omega_i$ :

\begin{equation}
	\begin{aligned}
		\omega_1 &= \int_1^3 \frac{(t-2)(t-3)}{(1-2)(1-3)} dt\\
		&= \frac{1}{2} \int_1^3 t^2 - 5t + 6 dt\\
		&= \frac{1}{2} \left [\frac{t^3}{3} - \frac{5t^2}{2} + 6t \right ]_1^3\\
		&= \frac{1}{2}  \left [\frac{9}{2} - \frac{23}{6} \right ] = \frac{1}{3}\\
		\omega_2 &= \int_1^3 \frac{(t-3)(t-3)}{(2-1)(2-3)} dt\\
		&= -1 \int_1^3 t^2 - 4t + 3 dt\\
		&= -1  \left [\frac{t^3}{3} - 2t^2 + 3t \right ]_1^3\\
		&= -1  \left [ 0 - \frac{4}{3} \right ] = \frac{4}{3}\\
		\omega_3 &= \int_1^3 \frac{(t-1)(t-2)}{(3-1)(3-2)} dt\\
		&= \frac{1}{2} \int_1^3 t^2 - 3t + 2 dt\\
		&= \frac{1}{2}  \left [\frac{t^3}{3} - \frac{3t^2}{2} + 2t \right ]_1^3\\
		&= \frac{1}{2}  \left [ \frac{3}{2} - \frac{5}{6} \right ] = \frac{1}{3}\\
	\end{aligned}
\end{equation}

On peut alors remplacer les $\omega_i$ pour obtenir :

\begin{equation}
	\begin{aligned}
		\int_a^b f(x) &\approx f(a) \cdot \frac{b-a}{2} \cdot \frac{1}{3} + f(\frac{a+b}{2}) \cdot \frac{b-a}{2} \cdot \frac{4}{3} + f(b) \cdot \frac{b-a}{2} \cdot \frac{1}{3} \\
		&= \frac{b-a}{6} \left (f(a) + 4f(\frac{a+b}{2})+f(b) \right )
	\end{aligned}
\end{equation}

\subsubsection{Point 2}

// TO DO

\subsection{Question 2}

\subsubsection{Point 1}

// TO DO

\subsubsection{Point 2}

// TO DO

\subsubsection{Point 3}

// TO DO


\subsection{Question 3}

\subsubsection{Newton-Cotes with $n = 2$}

\begin{equation}
	\begin{aligned}
		\int_{-1}^2 e^x dx &\approx \sum_{i=1}^2 f(x^{(i)}) \cdot h \cdot \int_1^2\frac{\prod_{j \neq i}(t-j)}{\prod_{j \neq i} (i - j)} dt\\
		&= f(x^{(1)}) \cdot h \cdot \omega_1 + f(x^{(2)}) \cdot h \cdot \omega_2
	\end{aligned}
\end{equation}

On a que :

\begin{equation}
	\begin{aligned}
		n &= 2\\
		[a, b] &= [-1, 2]\\
		h &= \frac{b-a}{n-1} = 3
	\end{aligned}
\end{equation}

On peut calculer les $\omega_i$ :

\begin{equation}
	\begin{aligned}
		\omega_1 &= \int_1^2 \frac{(t-2)}{(1-2)} dt\\
		&= -\left [\frac{t^2}{2}-2t \right ]_1^2\\
		&= -\left [0 - (-3/2)\right] = \frac{-3}{2}\\
		\omega_2 &= \int_1^2 \frac{(t-1)}{(2-1)} dt\\
		&= \left [\frac{t^2}{2}-t \right ]_1^2\\
		&= \left [0 - (-1/2)\right] = \frac{1}{2}\\
	\end{aligned}
\end{equation}

On peut dès lors résoudre :

\begin{equation}
	\begin{aligned}
		\int_{-1}^2 e^x dx &\approx f(x^{(1)}) \cdot h \cdot \omega_1 + f(x^{(2)}) \cdot h \cdot \omega_2\\
		&= f(-1) \cdot 3 \cdot \frac{-3}{2} + f(2) \cdot 3 \cdot \frac{1}{2}\\
		&= e^{(-1)} \cdot 3 \cdot \frac{-3}{2} + e^2 \cdot 3 \cdot \frac{1}{2}\\
		&= 9.42812
	\end{aligned}
\end{equation}

L'erreur est donc de $9.42812 - 7.02118 = 2.40694$.

\subsubsection{Gauss-Legendre with $n = 2$}

// TO DO

\subsection{Question 4}

// TO DO


\subsection{Question 5}

// TO DO